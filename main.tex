\documentclass{scrbook}

\usepackage[ngerman]{babel}
\usepackage[utf8]{inputenc}
\usepackage{amssymb, amsmath, microtype}
\usepackage{dsfont, xspace}


%\title{Numerische Mathematik 2}
%\author{Stephan Körkel}
%\date{\today}

\newcommand\Bezier{B\'ezier\xspace}
\newcommand\obda{o.\,B.\,d.\,A.\xspace}
\newcommand\zB{z.\,B.\xspace}
\newcommand\st{s.\,t.\xspace}

\newcommand\mathstuff{\mathds}
\newcommand\bitm{\begin{itemize}}
\newcommand\eitm{\end{itemize}}

\newcommand\that{{\hat t}}
\newcommand\tn{{t_0}}

\newcommand\Chi{\mathcal{X}}
\newcommand\cov{\mathrm{cov}}
\newcommand\vwahr{v_{wahr}}
\newcommand{\BIGOP}[1]{\mathop{\mathchoice%
{\raise-0.22em\hbox{\huge $#1$}}%
{\raise-0.05em\hbox{\Large $#1$}}{\hbox{\large $#1$}}{#1}}}
\newcommand{\bigtimes}{\BIGOP{\times}}
% nur fuer Bigboxplus andere Korrekturen
\newcommand{\BIGboxplus}{\mathop{\mathchoice%
{\raise-0.35em\hbox{\huge $\boxplus$}}%
{\raise-0.15em\hbox{\Large $\boxplus$}}{\hbox{\large $\boxplus$}}{\boxplus}}}

\renewcommand\phi{\varphi}
%\newcommand\ellipsen{\circledast}
\newcommand\ellipsen{\diamondsuit}
\newcommand\bpm{\begin{pmatrix}}
\newcommand\epm{\end{pmatrix}}
\newcommand\epsmach{\varepsilon_{mach}}
\newcommand\diag{\mathrm{diag}}
\newcommand\ov{\overline}
\newcommand\orthogonal{\perp}
\newcommand\Rang{\mathrm{Rg}}
\newcommand\Range{\mathrm{Range}}
\newcommand\trace{\mathrm{Sp}}
\newcommand\Image{\mathrm{Bild}}
\newcommand\Rg{\mathrm{Rg}}
\newcommand \tend{{t_{end}}}
\newcommand\Log{\mathrm{Log}}
\newcommand\R{\mathstuff{R}}
\newcommand\C{\mathstuff{C}}
\newcommand\E{\mathstuff{E}}
\newcommand\N{\mathstuff{N}}
\newcommand\Z{\mathstuff{Z}}
\newcommand\K{\mathstuff{K}}
\newcommand\Ho{\mathstuff{H}}
\newcommand\re{\mathrm{Re\,}}
\newcommand\im{\mathrm{Im\,}}
\newcommand\argmax{\arg\!\max}
\newcommand\argmin{\arg\!\min}
\renewcommand\l{\left}
\renewcommand\r{\right}
\newcommand\dt{\,\mathrm dt}
\newcommand\dd{\,\mathrm d}
\newcommand\RA{\Rightarrow}
\newcommand\LA{\Leftarrow}
\newcommand\LRA{\Leftrightarrow}
\newcommand\la\langle
\newcommand\ra\rangle
\newcommand\End{\mathrm{End}}
\newcommand\intl{\int\limits}
\newcommand\sign{\mathrm{sgn}}

\newcommand\zb{z.\,B.\xspace}

\newcommand\empha{\emph}

\newcommand\eps{\varepsilon}

\newcommand\chaptr[1]{\chapter*{#1} \addcontentsline{toc}{chapter}{#1}}

\newcommand\msection[1]{\section*{#1} \addcontentsline{toc}{section}{#1}}

\newcommand\msubsection[1]{\subsection*{#1} \addcontentsline{toc}{subsection}{#1}}
%\newcommand\msubsubsection[1]{\subsection*{#1} \addcontentsline{toc}{subsection}{:: #1}}

\newcommand\bemerkung[1]{\subsection*{Bemerkung #1} \addcontentsline{toc}{subsection}{Bemerkung #1}}

\newcommand\msubsubsection[1]{\subsubsection*{#1}}

\begin{document}






Gegeben Vektoren $a_1, \cdots, a_n \in \R^c$ und $b_1, \cdots, b_n \in \R^d$ suchen wir eine lineare Abbildung $X \colon \R^c \to \R^d$, die folgendes Zielfunktional minimiert:
%
\begin{align*}
f(X) :&= \sum\limits_{i=1}^n \|X a_i - b_i\|
\end{align*}
%
Wobei $\|\cdot\|$ entweder die 1-Norm oder das Quadrat der 2-Norm ist.
Notation:
%
\begin{align*}
X &= \begin{pmatrix} x_1^T \\ x_2^T \\ \vdots \\ x_d^T \end{pmatrix}
\end{align*}
%
Wobei $x_i \in \R^c$ Spaltenvektoren sind.

Die Normen sind invariant unter Transposition, deshalb können wir schreiben:
%
\begin{align*}
f(X) &= \sum\limits_{i=1}^n \|a_i^T X^T - b_i^T\|
\end{align*}
%
Nun könnten wir $X^T$ durch irgendein Least-Squares-Verfahren bestimmen lassen.
Das Problem daran ist, dass wir im Fall der 1-Norm zwar ein Verfahren für
$\argmin_u \|Yu-v\|_1$ mit $u \in \R^e$ haben aber keins für
$\argmin_U \|YU-V\|_1$ mit $U \in \R^{f \times g}$.
Deshalb brauchen wir eine modifizierte Version des Problems, bei dem die gesuchte Variable
als Spaltenvektor dargestellt ist.
Betrachte folgendes System:
%
\begin{align*}
\underbrace{
\begin{pmatrix}
a_1^T & 0 & \cdots & 0 \\
0 & a_1^T & 0 & 0 \\
\vdots & \ddots & \ddots & 0 \\
0 & \cdots & \cdots & a_1^T \\
\end{pmatrix}
}_{\tilde A_1}
\underbrace{
\begin{pmatrix}
x_1 \\ x_2 \\ \vdots \\ x_d
\end{pmatrix}
}_{\tilde X}
-
b_1
\end{align*}
%
Man beachte, dass $\tilde X$ ein Zeilenvektor mit $c\cdot d$ Einträgen ist.
$\tilde A_1$ ist eine $d \times c \cdot d$-Matrix, hat also $c \cdot d^2$ Einträge,
das macht keinen Spaß in Dimensionen größer drei, zum Farbfilter lernen geht es aber noch.

Wir berechnen $\|\tilde A_1 \tilde X -b\|$:
%
\begin{align*}
\|\tilde A_1 \tilde X -b\| &= 
\l\|
\begin{pmatrix}
a_1^T & 0 & \cdots & 0 \\
0 & a_1^T & 0 & 0 \\
\vdots & \ddots & \ddots & 0 \\
0 & \cdots & \cdots & a_1^T \\
\end{pmatrix}
\begin{pmatrix}
x_1 \\ x_2 \\ \vdots \\ x_d
\end{pmatrix}
-
b_1
\r\| \\
&= \l\| \begin{pmatrix} a_i^T x_1 \\ a_1^T x_2 \\ \cdots \\ a_1^T x_d \end{pmatrix} - b_1 \r\| \\
&= \l\| a_1^T X^T - b_1 \r\|
\end{align*}
%
Beide potentielle Normen sind "`separabel"' in dem Sinn, dass:
%
\begin{align*}
\|x\| &= \sum\limits_{i=1}^d \|x^i\|
\end{align*}
%
Wobei $x = (x_1, x_2, \cdots, x_i)^T \in \R^d$ ein beliebiger Vektor mit Einträgen $x^i$ ist.
Das erlaubt es uns, unsere Zielfunktion $f$ wie folgt zu schreiben:
%
\begin{align*}
f(X) &= \l\| \begin{pmatrix}
a_1^T & 0 & \cdots & 0 \\
0 & a_1^T & 0 & 0 \\
\vdots & \ddots & \ddots & 0 \\
0 & \cdots & \cdots & a_1^T \\
a_2^T & 0 & \cdots & 0 \\
0 & a_2^T & 0 & 0 \\
\vdots & \ddots & \ddots & 0 \\
0 & \cdots & \cdots & a_2^T \\
\vdots & \vdots & \vdots & \vdots \\
a_n^T & 0 & \cdots & 0 \\
0 & a_n^T & 0 & 0 \\
\vdots & \ddots & \ddots & 0 \\
0 & \cdots & \cdots & a_n^T \\
\end{pmatrix}
\begin{pmatrix}
x_1 \\ x_2 \\ \vdots \\ x_d
\end{pmatrix}
-
\begin{pmatrix}
b_1 \\ b_2 \\ \vdots \\ b_n
\end{pmatrix}
 \r\|
\end{align*}
%
Damit haben wir das Problem äquivalent umformuliert so dass die Einträge eines Zeilenvektors die gesuchten Variablen sind, aus denen wir am Ende die Matrix zusammenbauen können.

\end{document}
